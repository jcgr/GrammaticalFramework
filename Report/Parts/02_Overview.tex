\chapter{Overview}
\label{02}

As described in section \ref{01_01}, the legal text is short, to the point and not that good for translating directly into source code. An example of the legal text can be seen here:

	\centered{\formula{“If a candidate reaches the quota, he is declared elected.”}}

This piece of text shows the lack of detail in legal language: What is the quota? How do we ensure the candidate has reached the quota? What do we do with the ballot that makes him reach the quota? What happens with him when he is elected? Can he be elected even if there are no open seats? This is a lot of questions that the legal text does not answer. This means that assumptions have to be made and they are best avoided when it comes to voting.

The legal text do not provide much proof that it is correct either. This is something that will have to be done seperately and is called "searching for proof". Searching for proofs of this law will invariably result in a logical formula. 

In this project, we will focus on the single-transferrable voting protocol (STV) and as a running example, I will be using the following decleration from the "Linear Logical Voting Protocols"\cite{Deyoung11} paper by \citeauthor{Deyoung11}, which describes a candidate reaching the quota.

\begin{textoform}
	count/2: \\
	\formula{count-ballots(S, H, U) $\tensor$ \\
	uncounted-ballot(C, L) $\tensor$ hopeful(C, N) $\tensor$ \\
	$\bang$quota(Q) $\tensor$ $\bang$(N+1 $\ge$ Q) $\tensor$ winners(W) $\tensor$ \\
	$\bang$(S-1 $>$ 0) \\
	\hspace{5pt}$\lolli$ \{ counted-ballot(C, L) $\tensor$ $\bang$elected(C) $\tensor$ \\
	\hspace{29pt}winners([C $|$ W]) $\tensor$ count-ballots(S-1, H-1, U-1) \} }
\end{textoform}

In contrast, we will provide an example from the Votail\footnote{https://github.com/demtech/votail} system written in Java by Dermot Cochran and Joseph Kiniry.

We will notice that the method operates on another level of abstraction, and requires ambient knowledge regarding the methods and fields used (some of which are declared elsewhere), and that the programmer and reader understand each other. It also requires knowledge about the different semantics in Java, such as the keyword final, side effects of the code, shared memory and concurrency.

\begin{lstownjava}
  /**
* Elect any candidate with a quota or more of votes.
*/
  /*@ requires state == COUNTING;
@ assignable candidateList, ballotsToCount, candidates,
@ numberOfCandidatesElected, totalRemainingSeats;
@ assignable countStatus;
@ ensures countStatus.substate == AbstractCountStatus.CANDIDATE_ELECTED ||
@ countStatus.substate == AbstractCountStatus.SURPLUS_AVAILABLE;
@*/
  protected void electCandidatesWithSurplus() {
    while (candidatesWithQuota()
        && countNumberValue < CountConfiguration.MAXCOUNT
        && getNumberContinuing() > totalRemainingSeats) {
      
      updateCountStatus(AbstractCountStatus.CANDIDATES_HAVE_QUOTA);
      final int winner = findHighestCandidate();
      
      // Elect highest continuing candidate
      updateCountStatus(AbstractCountStatus.CANDIDATE_ELECTED);
      //@ assert 0 <= winner && winner < totalCandidates;
      //@ assert candidateList[winner].getStatus() == Candidate.CONTINUING;
      //@ assert numberElected < seats;
      //@ assert 0 < remainingSeats;
      /*@ assert (hasQuota(candidateList[winner]))
@ || (winner == findHighestCandidate())
@ || (getNumberContinuing() == totalRemainingSeats);
@*/
      electCandidate(winner);
      if (0 < getSurplus(candidates[winner])) {
        updateCountStatus(AbstractCountStatus.SURPLUS_AVAILABLE);
        distributeSurplus(winner);
      }
      
    }
  }
\end{lstownjava}

Furthermore, we will realize that is it a complicated process to translate the code into a natural language because of the afore-mentioned challenges. The main problem is the level of abstraction, as a verbalization system needs to abstract from the implementation details in order to generate readable and informative text.

Comparing the declaritive to the imperative formulation, we will notice that linear logic gives us the tools needed to express that a candidate cannot become unelected (the exclamation mark, which will be described later), and that it is more concise. The Java code assumes that the votes were already counted correctly, where the logical formula gives us a way to express how to count it.

Throughout the project, I will show that it is possible to write a program, in Grammatical Framework, that can translate logical formulas into English. The output should be understandable and make sense. Furthermore, it should be possible to name the arguments used in the formulas, as well as specify what the arguments used in the formula represents.