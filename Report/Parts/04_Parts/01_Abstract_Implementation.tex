\section{Constructing  the Abstract Syntax}
\label{04_01}

The first step in writing the program, was to construct the abstract syntax. The abstract syntax determines how the parts of the "language" (linear logic in this case) are put together to form sentences. It is important to make sure the abstract syntax is well constructed, or strange things may happen in the program.

It is, therefore, a good thing that we split the connectives of linear logic up in section \ref{03_01_02}. The resulting types can be translated almost directly into an abstract tree, which means we have a basic syntax already.

\lstCode{The first abstract syntax.}{04_01_C01}
\begin{lstgf}
abstract Laws = {
    
    flags startcat = Logic ;

    cat
        Logic ; Neg ; Pos ; Lolli ; Bang ; Atomic ; Conj ;

    fun
        Formular : Neg -> Logic ;

        -- Pos
        _Atom : Atomic -> Pos ;                 -- Turning an atomic into a positive type
        _Bang : Bang -> Atomic -> Pos ;         -- Using the unrestricted modality
        _Conj : Pos -> Conj -> Pos -> Pos ;     -- Using the simultaneous conjunction
        _Unit : Neg -> Pos ;                    -- Turns a negative into a positve
        _MPos : Pos -> Pos -> Pos ;             -- Attaches multiple positives to each other

        -- Neg
        _Pi : ArgColl -> Neg ;                  -- The universal quantification
        _Lolli : Pos -> Lolli -> Neg -> Neg ;   -- Using the linear implication
        _Mon : Pos -> Neg ;                     -- Turning a positive into a negative

        _Conj2 : Conj ;                         -- Simultaneous conjunction
        _Lolli2 : Lolli ;                       -- Linear implication
        _Bang2 : Bang ;                         -- Unrestricted modality
}
\end{lstgf}
Going through the abstract syntax, one will notice two things that do not stem from the connectives: There is something called an \cf{"Atomic"} (line 13) and there is an \cf{ArgColl} at the \cf{\_Pi}.The \cf{ArgColl} will be explained later. The \cf{Atomic} is used to represent "variables/functions(?)" along with the arguments they take. They will be explained in more detail later in this section. \\
There are two other things that needs an explanation as well. The \cf{\_Unit} turns a negative type into a positive type (it is actually a positive type, so there is no cheating here). The \cf{\_MPos} allows for multiple positive types to follow each other without connectives and is mainly used for allowing multiple universan quantifiers to follow each other.

The abstract syntax in \refCode{04_01_C01} introduced the \cf{Atomic}. The \cf{Atomic} needs an explanation and it needs to be defined as a function before the rest of the syntax can be written. An \cf{Atomic} represents a variable/function(?) in linear logic. This could be the \cf{voting-auth-card} or the \cf{blank-ballot} mentioned in section \ref{03_01_01}. An \cf{Atomic} can also represent a variable/function(?) that takes parameters, such as the ones in the example in section \ref{02_01}. \\
Looking at the example in section \ref{02_01}, one will notice that the formula not only contains variables/functions(?), but that it also contains arithmetic and inequality operations (for example  $\bang$(N + 1 $<$ Q)). This is in place of a function/variable(?), so the \cf{Atomic} needs to be able to represent that as well. \\
Putting that together, it means we need two kinds of atomics, one for variables/functions(?) and one for mathematical operations. The \cf{Atomic} is therefore represented in the following way in the abstract syntax.

\lstCode{Defining the \cf{Atomic} in the abstract syntax.}{04_01_C02}
\begin{lstgf}
    cat
        Atomic ; Ident ; MathFormula ;

    fun
        -- Atomic
        Atom_Ident : Ident -> Atomic ;          -- Represents the atomic variables/functions
        Atom_Math : MathFormula -> Atomic ;     -- Represents the mathematical operations
\end{lstgf}

Now we have the \cf{Atomic} defined, but we have introduced two new categories at the same time; \cf{Ident} and \cf{MathFormula}. The \cf{Ident} represents the variables through the variable/function(?) name and the arguments it takes. To be able to use it, we need to define not just the \cf{Ident}, but also the arguments needed. The abstract syntax is once again extended.

\lstCode{Defining the \cf{Ident} and the arguments it needs.}{04_01_C03}
\begin{lstgf}
    cat
        Ident ; Arg ; ArgColl ;

    fun
        -- Identifiers
        Ident_Hopeful, Ident_Tally, Ident_BangElectAll, Ident_Elected, Ident_Defeated, Ident_Quota, Ident_Minimum,
        Ident_DefeatMin, Ident_Transfer, Ident_Counted, Ident_Uncounted, Ident_Winners, Ident_Begin : ArgColl ->
        ArgColl -> ArgColl -> ArgColl -> ArgColl -> ArgColl -> ArgColl -> ArgColl -> ArgColl -> Ident ;
        
        -- Arguments
        Arg_C, Arg_N, Arg_S, Arg_H, Arg_U, Arg_Q, Arg_L, Arg_M, Arg_W, Arg_0, Arg_1, Arg_Nil : Arg ;
        _Arg : Arg -> ArgColl ;
        _ArgPlus, _ArgMinus : ArgColl -> ArgColl ;
        _ArgListEmpty : ArgColl ;
        _ArgList : ArgColl -> ArgColl ->  ArgColl ;
\end{lstgf}

This looks confusing, but there is a method to the madness. Starting with the arguments, there are a lot with the name \cf{Arg\_something}. Each of them represents one of the arguments used by the different variables (S, N, H, etc.). The \cf{Arg}s have to be converted into \cf{ArgColl} to be used. This is done to make it possible to use lists of arguments along with Celf's notation of plus and minus. \\
Both \cf{\_ArgPlus} and \cf{\_ArgMinus} take an \cf{ArgColl} as argument and produces an \cf{ArgColl}. The reason behind this, is that a logical formula can say "(S - 1) - 1". If they took an \cf{Arg} as the argument, it would not be possible to represent such a case.\\
\cf{\_ArgListEmpty} is simply an \cf{ArgColl}. It does not take any arguments. \cf{\_ArgList}, on the other hand, takes two \cf{ArgColl}s to produce an \cf{ArgColl}. This way, it is possible to use \cf{\_ArgPlus}/\cf{\_ArgMinus} inside the list, which may or may not be needed.

Having understood the arguments, the identifiers are a bit simpler. Each variable used for the logical formulas concering laws (\note{Appendix with them?}) is represented by an \cf{Identifier}. Each \cf{Identifier} takes up to nine \cf{ArgColl}s and produces an \cf{Ident}. Nine arguments may seem excessive, but it is up to the concrete implementations to decide how many of the arguments that are used. An \cf{Ident} does not have to use all nine. \note{Explain better}

With the \cf{Ident}s in place (and thereby half the \cf{Atomic}s covered),  it is time to look at the other \cf{Atomic}: The one concerning mathematical formulas. The mathematical formulas needs arithemetic operations and inequality operations, so we extend the abstract syntax with the following:

\lstCode{Defining the mathematical operations.}{04_01_C04}
\begin{lstgf}
    cat
        ArgColl ; Math ; MathFormula ; ArithmeticOperation ; InequalityOperation ;

    fun
        -- Mathematic operations
        _FinalFormula : Math -> InequalityOperation -> Math -> MathFormula ;
        _Math : ArgColl -> Math ;
        _MathArgs : Math -> ArithmeticOperation -> Math -> Math ;

        _Division, _Addition, _Subtraction, _Multiplication : ArithmeticOperation ;
        Greater, GreaterEqual, Equal, LessEqual, Less : InequalityOperation ;
\end{lstgf}

All mathematical formulas in the logical formulas have some sort of inequality operation. Therefore, \cf{\_FinalFormula} is the only \cf{MathFormula}, and it is made from a \cf{Math}, an \cf{InequalityOperation} and a second \cf{Math}.\\
\cf{\_Math} simply takes an \cf{ArgColl} and produces a \cf{Math}. This can be used either for the \cf{\_FinalFormula} directly, or for the \cf{\_MathArgs} that takes two \cf{Math}s and a \cf{ArithmeticOperation} and returns a \cf{Math}. That way, arithmetic operations can be either a simple operation, or the \cf{Math}(s) it is made of can be made \cf{\_MathArgs}.\\
The \cf{InequalityOperation} and \cf{ArithmeticOperation} are simple and need no futher explanation.

\note{Some sort of ending}. \\
The full abstract implementation can be found in appendix \ref{A_01}.