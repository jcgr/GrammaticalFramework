\section{Running the Program}
\label{04_04}

In this section, I will show the result of parsing a logical formula into the program.

\subsection{The example}

The formula to be run is the one from the example in chapter \ref{02}. As the program has been written to accept Celf's output, I will use the version of the formula from the end of section \ref{03_01_03}. Due to how GF works, I have had to add extra spaces to the formula, but the reader should be able to verify that it is the same:

\begin{textoform}
	\formula{Pi C : candidate . Pi H : nat . Pi L : list . Pi N : nat . \\
		Pi Q : nat . Pi S : nat . Pi U : nat . Pi W : list .\\
		count-ballots ( s ! ( s ! S ) ) ( s ! H ) ( s ! U ) *\\
          	uncounted-ballot C L *\\
          	hopeful C N *\\
          	!quota Q *\\
          	!nat-lesseq Q ( s ! N ) *v
          	winners W\\
          	\hspace{10pt}-o \{counted-ballot C L *\\
          	\hspace{30pt}!elected C *\\
          	\hspace{30pt}winners ( cons ! C ! W ) *\\
          	\hspace{30pt}count-ballots ( s ! S ) H U\}. }
\end{textoform}

With the formula changed to fit the syntax of the program, we will parse it and have it translated into English. We parse it in GF by writing
\begin{texto2}
p -lang=LawsLin "Pi C : candidate . Pi H : nat . Pi L : list . Pi N : nat . Pi Q : nat . Pi S : nat . Pi U : nat . Pi W : list . count-ballots ( s ! ( s ! S ) ) ( s ! H ) ( s ! U ) * uncounted-ballot C L * hopeful C N * !quota Q * !nat-lesseq Q ( s ! N ) * winners W -o \{ counted-ballot C L * !elected C * winners ( cons ! C ! W ) * count-ballots ( s ! S ) H U \}" $|$ l -treebank
\end{texto2}

What this does, is to tell GF that we want to parse it as the language defined by the -lang flag. After it has been parsed, we tell GF to linearize it  and use the -treebank flag to have it linearized in all the languages that have been loaded. The result is the following:

\begin{texto2}
LawsEng: \\
C is a candidate . H is a natural number . \\
L is a list . N is a natural number . \\
Q is a natural number . S is a natural number . \\
U is a natural number . W is a list .\\\\
If we are counting votes and there are ( ( S - 1 ) - 1 ) seats open, \\
( H - 1 ) hopefuls, and ( U - 1 ) uncounted votes in play , and \\
there is an uncounted vote with highest preference for candidate C with a list L \\
of lower preferences , and \\
candidate C is a hopeful with N votes , and Q votes are needed to be elected , and \\
( Q is less than or equal to ( N - 1 ) ) , \\
and the candidates in the list W have been elected so far \\
then \{ there is a counted vote with highest preference for candidate C with a \\
list L of lower preferences , and \\
candidate C has been elected , and \\
the candidates in the list consisting of C and W have been elected so far , and \\
we are counting votes and there are ( S - 1 ) seats open, H hopefuls, and \\
U uncounted votes in play \}
\end{texto2}

In the output, one will notice that there are cases where a sentence feels a bit odd, for example the last sentence, where it would make more sense if it said that the seats, hopefuls and uncounted votes are what is remaining. This is due to a limitation in the program (more about that in section \ref{04_04_02}), which makes the program uses the same sentences for \cf{Ident}s, no matter what side of the $\lolli$ they are on.

Though some sentences may seem strange, the output is nonetheless in sensible English. The arguments are explained, so there is no confusion about what they are, and the different identifiers have been translated into meaningful sentences. The output is a lot more explanatory than the legal text and is something a lot of people would be able to understand.