\section{Example}
\label{02_01}

As described in section \ref{01_01}, the legal text is short, to the point and not that good for translating directly into source code. An example of the legal text can be seen here:

	\centered{\formula{“If a candidate reaches the quota, he is declared elected.”}}

This piece of text shows the lack of detail in legal language: How do we ensure the candidate has reached the quota? What is the quota? What do we do with the ballot that makes him reach the quota even? What happens with him when he is elected? Can he be elected even if there are no open seats? The latter is a very important question, one that is not answered by the legal text. Common sense dictates that the answer is “no”, but that is an assumption one has to make and assumptions are best avoided when it comes to voting.  \note{I do not understand comment 4.5}

Searching for proofs of this law will invariably result in a logical formula. As a running example, I will be using the following decleration from the "Linear Logical Voting Protocols"\cite{Deyoung11} paper by \citeauthor{Deyoung11}, which describes a candidate reaching the quota.

\begin{textoform}
	count/2: \\
	\formula{count-ballots(S, H, U) $\tensor$ \\
	uncounted-ballot(C, L) $\tensor$ hopeful(C, N) $\tensor$ \\
	$\bang$quota(Q) $\tensor$ $\bang$(N+1 $\ge$ Q) $\tensor$ winners(W) $\tensor$ \\
	$\bang$(S-1 $>$ 0) \\
	\hspace{5pt}$\lolli$ \{ counted-ballot(C, L) $\tensor$ $\bang$elected(C) $\tensor$ \\
	\hspace{29pt}winners([C $|$ W]) $\tensor$ count-ballots(S-1, H-1, U-1) \} }
\end{textoform}

This formula accurately describes the process for checking if a candidate reaches the quota and then marking him as elected.


% Understanding this formula requires knowledge of how linear logic works and is therefore not suited for the legal text.

% In contrast, we give an example from Dermot...  a method programmed
% in Java that marks a candidate as elected.

% We notice that the method operates on a different level of
% abstraction and requires ambient knowledge regarding the meaning of
% all methods and fields used in the function but declared elsewhere,
% state of mind, common understanding between programmer and reader,
% and most importantly, the operational semantics of Java, including
% the meaning of final, side effects, shared memory, and concurrency.

\begin{lstownjava}
  /**
* Elect any candidate with a quota or more of votes.
*/
  /*@ requires state == COUNTING;
@ assignable candidateList, ballotsToCount, candidates,
@ numberOfCandidatesElected, totalRemainingSeats;
@ assignable countStatus;
@ ensures countStatus.substate == AbstractCountStatus.CANDIDATE_ELECTED ||
@ countStatus.substate == AbstractCountStatus.SURPLUS_AVAILABLE;
@*/
  protected void electCandidatesWithSurplus() {
    while (candidatesWithQuota()
        && countNumberValue < CountConfiguration.MAXCOUNT
        && getNumberContinuing() > totalRemainingSeats) {
      
      updateCountStatus(AbstractCountStatus.CANDIDATES_HAVE_QUOTA);
      final int winner = findHighestCandidate();
      
      // Elect highest continuing candidate
      updateCountStatus(AbstractCountStatus.CANDIDATE_ELECTED);
      //@ assert 0 <= winner && winner < totalCandidates;
      //@ assert candidateList[winner].getStatus() == Candidate.CONTINUING;
      //@ assert numberElected < seats;
      //@ assert 0 < remainingSeats;
      /*@ assert (hasQuota(candidateList[winner]))
@ || (winner == findHighestCandidate())
@ || (getNumberContinuing() == totalRemainingSeats);
@*/
      electCandidate(winner);
      if (0 < getSurplus(candidates[winner])) {
        updateCountStatus(AbstractCountStatus.SURPLUS_AVAILABLE);
        distributeSurplus(winner);
      }
      
    }
  }
\end{lstownjava}

% Furthermore, we observe that it is very difficult to translate this into natural language, because of the aforementioned challenges. 
% We see the main proble is the level abstraction, because any verbalization system needs to abstract from the implementation details 
% in order to generate readable and informative text. 

% If we compare teh declartive to the imperative formulation, we notice that linear logic gives us the tools we need to express that a candidate cannot become unelected (see bang)
% and that is also more concise.  The Java code, for example assumes that the votes were already counted correctly, wheres the logical formulations gives us a way to express that an
% uncounted ballot becomes counted.


 