\section{Example}
\label{02_01}

As described in section \ref{01_01}, the legal text is short, to the point and not that good for translating directly into source code. An example of the legal text can be seen here:

	\centered{\formula{“If a candidate reaches the quota, he is declared elected.”}}

This piece of text shows the lack of detail in legal language: How do we ensure the candidate has reached the quota? What is the quota? What do we do with the ballot that makes him reach the quota even? What happens with him when he is elected? Can he be elected even if there are no open seats? The latter is a very important question, one that is not answered by the legal text. Common sense dictates that the answer is “no”, but that is an assumption one has to make and assumptions are best avoided when it comes to voting.  \note{I do not understand comment 4.5}

Searching for proofs of this law will invariably result in a logical formula. As a running example, I will be using the following decleration from the "Linear Logical Voting Protocols"\cite{Deyoung11} paper by \citeauthor{Deyoung11}, which describes a candidate reaching the quota.

\begin{textoform}
	count/2: \\
	\formula{count-ballots(S, H, U) $\tensor$ \\
	uncounted-ballot(C, L) $\tensor$ hopeful(C, N) $\tensor$ \\
	$\bang$quota(Q) $\tensor$ $\bang$(N+1 $\ge$ Q) $\tensor$ winners(W) $\tensor$ \\
	$\bang$(S-1 $>$ 0) \\
	\hspace{5pt}$\lolli$ \{ counted-ballot(C, L) $\tensor$ $\bang$elected(C) $\tensor$ \\
	\hspace{29pt}winners([C $|$ W]) $\tensor$ count-ballots(S-1, H-1, U-1) \} }
\end{textoform}

This formula accurately describes the entire process involved in checking if a candidate reaches the quota and then marking him as elected. Understanding this formula requires knowledge of how linear logic works and is therefore not suited for the legal text.

\begin{lstownjava}
 /**
   * Get next preference candidate ID
   * 
   * @param offset
   *          The number of preferences to look ahead
   * @return The next preference candidate ID
   */
  /*@ public normal_behavior
    @   requires 0 <= positionInList + offset;
    @   ensures (\result == NONTRANSFERABLE)
    @     || (\result == getPreference(positionInList + offset));
    @*/
  public/*@ pure @*/int getNextPreference(final int offset) {
    final int index = positionInList + offset;
    if (index < numberOfPreferences && index < preferenceList.length) {
      return preferenceList[index];
    }
    return NONTRANSFERABLE;
  }
\end{lstownjava}
